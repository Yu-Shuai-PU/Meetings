% Created 2025-10-01 Wed 23:41
% Intended LaTeX compiler: pdflatex
\documentclass[presentation]{beamer}
\usepackage{graphicx}
\usepackage{longtable}
\usepackage{wrapfig}
\usepackage{rotating}
\usepackage[normalem]{ulem}
\usepackage{amsmath}
\usepackage{amssymb}
\usepackage{capt-of}
\usepackage{hyperref}
\usepackage{fontspec}
\usepackage{onimage}
\usecolortheme{princeton}
\usetheme{Madrid}
\setbeameroption{show notes on second screen}
\author{Yu Shuai}
\date{\today}
\title{Non-intrusive model reduction of shift-equivariant systems via data-driven projection and reduced dynamics}
\hypersetup{
 pdfauthor={Yu Shuai},
 pdftitle={Non-intrusive model reduction of shift-equivariant systems via data-driven projection and reduced dynamics},
 pdfkeywords={},
 pdfsubject={},
 pdfcreator={Emacs 29.4 (Org mode 9.6.15)}, 
 pdflang={English}}
\begin{document}

\maketitle
\begin{frame}{Contents}
\tableofcontents[hideallsubsections]
\end{frame}

\section{Before we start: SIAM meeting attendance? NWCS26 or AN26}
\label{sec:org2b20211}
\begin{frame}{Contents}
\tableofcontents[currentsection, hideothersubsections]
\end{frame}


\section{Method}
\label{sec:orgaef6232}
\begin{frame}{Contents}
\tableofcontents[currentsection, hideothersubsections]
\end{frame}

\subsection{Symmetry-reduced non-intrusive trajectory-based ROM (SR-NiTROM)}
\label{sec:org6c4f9ae}
\begin{frame}[label={sec:orgd436b07}]{Formulation of the SR-NiTROM}
\begin{itemize}[<+->]
\item Consider a shift-equivariant FOM with traveling solutions:
\begin{equation}
   \label{eq:1}
   u_{t} = f(u), \qquad u(x, t) = \hat{u}(x - c(t), t).
\end{equation}
\item Suppose we have a collection of training snapshots \(\{u(t_{m})\}_{m = 0}^{N_{t}-1}\), \(u(t)\in\mathbb{R}^{N}\).
\item We seek to find \(\Phi, \Psi\in\mathbb{R}^{n\times r}\), such that \(\Phi(\Psi^{\top}\Phi)^{-1}\Psi^{\top}\) is a projection.
\item From this projection operator, we can encode the FOM state with a low-dim representation \(a(t)\in\mathbb{R}^{r}\):
\begin{align}
  \label{eq:2}
    a &= \Psi^{\top}u \notag\\
    \hat{u}_{r} &= \Phi(\Psi^{\top}\Phi)^{-1}a.
\end{align}
\item The ROM dynamics is given in a symmetry-reduced form:
\begin{subequations}
  \label{eq:3}
  \begin{align}
    \label{eq:3a}
    \dot{a}_{i} &= A_{ij}a_{j} + B_{ijk}a_{j}a_{k} + \dot{c} M_{ij}a_{j}\\
    \label{eq:3b}
    \dot{c} &= -\frac{p_{i}a_{i} + Q_{ij}a_{i}a_{j}}{s_{i}a_{i}}\\
    \label{eq:3c}
    M &= \Psi^{\top}\partial_{x}\Phi(\Psi^{\top}\Phi)^{-1}, \quad s = \langle\partial_{x}\Phi(\Psi^{\top}\Phi)^{-1}, \partial_{x}u_{0}\rangle
  \end{align}
\end{subequations}
\end{itemize}
\end{frame}

\begin{frame}[label={sec:org7d0e97e}]{The optimization problem of SR-NiTROM}
\begin{itemize}[<+->]
\item The \textcolor{blue}{trajectory-based} objective function:
\begin{equation}
  \label{eq:4}
  J = \sum_{m = 0}^{N_{t} - 1}\|\hat{u}_{r}(t_{m}) - \hat{u}(t_{m})\|^{2} + \beta(c_{r}(t_{m}) - c(t_{m}))^{2}.
\end{equation}
\begin{itemize}
\item \(\beta = \gamma\sum_{m}\|\hat{u}(t_{m})\|^{2}/\sum_{m}(c(t_{0}) - c(t_{m}))^{2}\): relative weights.
\item \(\gamma\): hyperparameter.
\end{itemize}
\item The unconstrained Lagrangian with multipliers:
\begin{align}
  \label{eq:5}
  L &= \sum_{m = 0}^{N_{t} - 1}\bigg(\|\hat{u}_{r}(t_{m}) - \hat{u}(t_{m})\|^{2} + \beta(c_{r}(t_{m}) - c(t_{m}))\notag\\
    &+ \int_{t_{0}}^{t_{m}}\lambda_{m}^{\top}(\dot{a} - Aa - B(a,a) - \dot{c}Ma)\mathrm{d}t\\
    &+ \int_{t_{0}}^{t_{m}}\mu_{m}(\dot{c} + \frac{p_{i}a_{i} + Q_{ij}a_{i}a_{j}}{s_{i}a_{i}})\mathrm{d}t\\
    &+ \lambda_{m}(t_{0})(a(t_{0}) - \Psi^{\top}\hat{u}(t_{0}))\bigg), \quad \lambda_{m}\in\mathbb{R}^{r}, \mu_{m}\in\mathbb{R}.
\end{align}
\end{itemize}
\end{frame}

\section{Results}
\label{sec:org88a3cc1}
\begin{frame}{Contents}
\tableofcontents[currentsection, hideothersubsections]
\end{frame}

\subsection{Single trajectory: SR-NiTROM vs SR-Galerkin}
\label{sec:org6e7ee9f}
\begin{frame}[label={sec:orgb7ba668}]{Numerical details}
\begin{itemize}[<+->]
\item FOM: Kuramoto-Sivashinsky equation
\begin{equation}
  \label{eq:6}
  u_{t} = -uu_{x} - u_{xx} - \nu u_{xxxx}, \quad x\in[0, 2\pi].
\end{equation}
\begin{itemize}
\item \(\nu = 4/87\) for traveling-wave patterns.
\item Periodic BCs, \(N=40\) Fourier modes, \(\Delta t = 10^{-3}\).
\item Sample interval: 10 timesteps between 2 adjacent snapshots.
\end{itemize}

\item Optimization of the SR-NiTROM:
\begin{itemize}
\item coordinate-descent method
conjugate gradient optimizer for each subproblems.
\item 20 outer loops, 5 CG updates per outer loops.
\item Initial conditions: POD bases (capturing >99.5\% energy) + Galerkin-projected tensors.
(imitating the training result of the re-projected SR-OpInf ROM)
\end{itemize}
\end{itemize}
\end{frame}

\begin{frame}[label={sec:org979d34b}]{Results: single transient trajectory from t = 30 to t = 40}
\begin{itemize}[<+->]
\item Relative weight: 10.0, 4-dim ROM
\item FOM vs SR-Galerkin vs SR-NiTROM
\begin{figure}[tbp]
    \centering
    \begin{tikzonimage}[width=0.4\linewidth]{figures/sol_FOM.png}%[tsx/show help lines]
      \node at (0.45, -0.05) {(a) FOM};
    \end{tikzonimage}
    \begin{tikzonimage}[width=0.4\linewidth]{figures/sol_SRG.png}%[tsx/show help lines]
      \node at (0.45, -0.05) {(b) SR-Galerkin};
    \end{tikzonimage}
    \begin{tikzonimage}[width=0.4\linewidth]{figures/sol_SRN.png}%[tsx/show help lines]
      \node at (0.45, -0.05) {(c) SR-NiTROM};
    \end{tikzonimage}
    \begin{tikzonimage}[width=0.4\linewidth]{figures/shift_amount_SRN.png}%[tsx/show help lines]
      \node at (0.45, -0.05) {(d) shift amount};
    \end{tikzonimage}
    \label{fig:contours_single_traj_comparison_3_models}
  \end{figure}
\end{itemize}
\end{frame}

\subsection{Multiple trajectories: SR-NiTROM vs SR-Galerkin}
\label{sec:orga83c877}
\begin{frame}[label={sec:orgfa39b45}]{Results: multiple trajectories from perturbed initial conditions}
\begin{itemize}[<+->]
\item 9 trajectories, 7-dim ROM
\item Initial conditions: post-transient solution snapshot + perturbations
\(u(t = 80) + \{0, \sin(x), ..., \sin(4x), \cos(x), ..., \cos(4x)\}\).
\item Strategy A: 20 outer training loops (10 on bases, 10 on tensors), 5 CG updates per outer loop
\item Strategy B: 10 outer training loops on tensors only with fixed POD bases
\item Training loss: \textcolor{blue}{not too much difference on the training set}.
\begin{figure}[tbp]
    \centering
    \begin{tikzonimage}[width=0.4\linewidth]{figures/training_error_both.png}%[tsx/show help lines]
      \node at (0.45, -0.05) {(a) Strategy A};
    \end{tikzonimage}
    \begin{tikzonimage}[width=0.4\linewidth]{figures/training_error_tensors.png}%[tsx/show help lines]
      \node at (0.45, -0.05) {(b) Strategy B};
    \end{tikzonimage}
    \label{fig:training_loss_both_vs_tensors_only}
  \end{figure}
\end{itemize}
\end{frame}

\begin{frame}[label={sec:orgaa308a2}]{Results: multiple trajectories from perturbed initial conditions}
\begin{itemize}[<+->]
\item Trajectory 1:
\begin{figure}[tbp]
    \centering
    \begin{tikzonimage}[width=0.4\linewidth]{figures/sol_FOM_000.png}%[tsx/show help lines]
      \node at (0.45, -0.05) {(a) FOM};
    \end{tikzonimage}
    \begin{tikzonimage}[width=0.4\linewidth]{figures/sol_SRG_000.png}%[tsx/show help lines]
      \node at (0.45, -0.05) {(b) SR-Galerkin};
    \end{tikzonimage}
    \begin{tikzonimage}[width=0.4\linewidth]{figures/sol_SRN_000_both.png}%[tsx/show help lines]
      \node at (0.45, -0.05) {(c) SR-NiTROM A};
    \end{tikzonimage}
    \begin{tikzonimage}[width=0.4\linewidth]{figures/sol_SRN_000_tensors.png}%[tsx/show help lines]
      \node at (0.45, -0.05) {(d) SR-NiTROM B};
    \end{tikzonimage}
    \label{fig:contours_sol_000_different_models}
\end{figure}
\end{itemize}
\end{frame}

\begin{frame}[label={sec:org8ea52ba}]{Results: multiple trajectories from perturbed initial conditions}
\begin{itemize}[<+->]
\item Trajectory 1:
\begin{figure}[tbp]
    \centering
    \begin{tikzonimage}[width=0.4\linewidth]{figures/sol_FOM_fitted_000.png}%[tsx/show help lines]
      \node at (0.45, -0.05) {(a) FOM};
    \end{tikzonimage}
    \begin{tikzonimage}[width=0.4\linewidth]{figures/sol_SRG_fitted_000.png}%[tsx/show help lines]
      \node at (0.45, -0.05) {(b) SR-Galerkin};
    \end{tikzonimage}
    \begin{tikzonimage}[width=0.4\linewidth]{figures/sol_SRN_fitted_000_both.png}%[tsx/show help lines]
      \node at (0.45, -0.05) {(c) SR-NiTROM A};
    \end{tikzonimage}
    \begin{tikzonimage}[width=0.4\linewidth]{figures/sol_SRN_fitted_000_tensors.png}%[tsx/show help lines]
      \node at (0.45, -0.05) {(d) SR-NiTROM B};
    \end{tikzonimage}
    \label{fig:fitted_contours_sol_000_different_models}
\end{figure}
\end{itemize}
\end{frame}

\begin{frame}[label={sec:orgf68e9e7}]{Results: multiple trajectories from perturbed initial conditions}
\begin{itemize}[<+->]
\item Trajectory 1, shift amount and shift speed:
\begin{figure}[tbp]
    \centering
    \begin{tikzonimage}[width=0.4\linewidth]{figures/shift_amount_SRN_FOM_000.png}%[tsx/show help lines]
      \node at (0.45, -0.05) {(a) Shift amounts};
    \end{tikzonimage}
    \begin{tikzonimage}[width=0.4\linewidth]{figures/shift_speed_SRN_FOM_000.png}
      \node at (0.45, -0.05) {(a) Shift speeds};
    \end{tikzonimage}
    \label{fig:shifting_speeds_sol_000_different_models}
\end{figure}
\end{itemize}
\end{frame}

\begin{frame}[label={sec:org0029e83}]{Conclusions:}
\begin{itemize}[<+->]
\item For the reconstruction of a single training trajectory \textcolor{blue}{including transient}, SR-NiTROM outperforms SR-Galerkin ROM (and SR-OpInf of course).
\item For the reconstruction of multiple transient trajectories, we find that:
\begin{itemize}
\item SR-NiTROM gives better approximation of template-aligned snapshots and shift amounts than the SR-Galerkin ROM.
\item It's better to optimize both the bases and the tensors to minimize our loss function.
\item \textcolor{blue}{However, SR-NiTROM with only trained tensors and POD bases attains the least reconstruction error of the raw snapshots.}
\begin{itemize}
\item Why: our loss function = error in aligned snapshots + error in shift amounts, not error in raw snapshots.
\item Trade-off: we may want to switch to error in raw snapshots, but then the optimizer doesn't know if the error comes from mismatch of aligned profiles or shift amounts.
\item This new loss function is reasonable since small shift mismatch can lead to large error in raw snapshots.
\end{itemize}
\end{itemize}

\item To-dos: test our SR-NiTROM on unseen trajectories. Compute the obliqueness of projection.
\end{itemize}
\end{frame}
\end{document}
