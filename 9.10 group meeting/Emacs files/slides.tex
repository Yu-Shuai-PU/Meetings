% Created 2025-09-09 Tue 21:58
% Intended LaTeX compiler: pdflatex
\documentclass[presentation]{beamer}
\usepackage{graphicx}
\usepackage{longtable}
\usepackage{wrapfig}
\usepackage{rotating}
\usepackage[normalem]{ulem}
\usepackage{amsmath}
\usepackage{amssymb}
\usepackage{capt-of}
\usepackage{hyperref}
\usepackage{fontspec}
\usepackage{onimage}
\usecolortheme{princeton}
\usetheme{Madrid}
\setbeameroption{show notes on second screen}
\author{Yu Shuai}
\date{\today}
\title{Non-intrusive model reduction of shift-equivariant systems via data-driven projection and reduced dynamics}
\hypersetup{
 pdfauthor={Yu Shuai},
 pdftitle={Non-intrusive model reduction of shift-equivariant systems via data-driven projection and reduced dynamics},
 pdfkeywords={},
 pdfsubject={},
 pdfcreator={Emacs 29.4 (Org mode 9.6.15)}, 
 pdflang={English}}
\begin{document}

\maketitle
\begin{frame}{Contents}
\tableofcontents[hideallsubsections]
\end{frame}

\section{Review: Symmetry-reduced operator inference (SR-OpInf)}
\label{sec:org4c56df1}
\begin{frame}{Contents}
\tableofcontents[currentsection, hideothersubsections]
\end{frame}

\subsection{SR-OpInf for model reduction}
\label{sec:org61097ba}
\begin{frame}[label={sec:orgc3f05c2}]{SR-OpInf for model reduction}
\note{Here I use this diagram to show the algorithm of symmetry-reduced operator inference.
We start from a black-boxed full-order model partial u partial t = f(u).
The system is shift equivariant, means that you can move your solution, and it has traveling solutions.
We want to predict those traveling solutions uhat using a small number of modes. That's our goal.
To do this, we train a model. We first collect some data, and then perform template fitting to align those traveling solutions against a template to remove their drifting.
From this fitting, we can get the aligned solutions, their RHSs and the shifting amount.
Based on this, we utilize the reconstruction equation to compute their traveling speeds.
Once we collect these data, we now try to approximate the frozen solution profiles using POD modes, and assume a symmetry-reduced ROM based on the full-order dynamics of fitted solution uhat.
Finally, we learn the functions.}
\begin{itemize}[<+->]
\item \begin{figure}[tbp]
\centering
\begin{tikzonimage}[width=0.9\linewidth]{figures/fig1_illustration_sropinf.png}%[tsx/show help lines]
\end{tikzonimage}
  \vspace{-1em}
  \caption{Training procedure of the SR-OpInf ROM}
  \label{fig:1_illustration_sropinf}
\end{figure}
\vspace{-1em}
\item Drawbacks of the SR-OpInf algorithm
\begin{itemize}
\item Usage of \textcolor{blue}{orthogonal projection (insensitive to transient)}.
\item Derivative-based loss function \textcolor{blue}{(instead of errors in $u(t)$)}
\end{itemize}
\end{itemize}
\end{frame}


\begin{frame}[label={sec:org1809a1d}]{SR-OpInf for model reduction}
\begin{itemize}
\item \begin{figure}[tbp]
\centering
\begin{tikzonimage}[width=0.9\linewidth]{figures/fig1_illustration_sropinf.png}%[tsx/show help lines]
\end{tikzonimage}
  \vspace{-1em}
  \caption{Training procedure of the SR-OpInf ROM}
  \label{fig:1_illustration_sropinf}
\end{figure}
\vspace{-1em}
\item Potential improvement
\begin{itemize}
\item Adopt the \textcolor{blue}{non-intrusive trajectory-based optimization} of ROM (\textcolor{blue}{NiTROM}, Padovan et al. 2024) to train the projection and dynamics.
\end{itemize}
\end{itemize}
\end{frame}

\section{Background}
\label{sec:orgde72cc0}
\begin{frame}{Contents}
\tableofcontents[currentsection, hideothersubsections]
\end{frame}

\subsection{Introduction to NiTROM}
\label{sec:org888baa8}
\begin{frame}[label={sec:org78524fc}]{NiTROM: non-intrusive trajectory-based ROM}
\begin{itemize}[<+->]
\item Consider a general nonlinear FOM (no need to be shift-equivariant):
\begin{equation}
   \label{eq:1}
   \frac{\mathrm{d}u}{\mathrm{d}t} = f(u), u\in\mathbb{R}^{n}, u(0) = u_{0}
\end{equation}
\item Define an oblique projection with two matrices \(\Phi, \Psi\in\mathbb{R}^{n\times r}\)
\begin{subequations}
\begin{align}
  \label{eq:2}
  a &= \Psi^{\top}u\in\mathbb{R}^{r}\\
  u &\approx \Phi(\Psi^{\top}\Phi)^{-1}a
\end{align}
\end{subequations}
\begin{itemize}
\item If \(\Phi = \Psi\), then the projection is orthogonal (e.g., POD basis matrix)
\end{itemize}
\item The non-intrusive ROM then takes the form of
\begin{subequations}
\begin{align}
  \label{eq:3}
  \frac{\mathrm{d}a}{\mathrm{d}t} &= g(a), a(0) = \Psi^{\top}u(0)\\
  u &\approx u_{r} = \Phi(\Psi^{\top}\Phi)^{-1}a
\end{align}
\end{subequations}
\begin{itemize}
\item \(g(a) = Aa + B:(aa^{\top}) = A_{ij}a_{j} + B_{ijk}a_{j}a_{k}, A\in\mathbb{R}^{r\times r}, B\in\mathbb{R}^{r\times r\times r}\).
\end{itemize}
\end{itemize}
\end{frame}

\begin{frame}[label={sec:orgd97de61}]{Trajectory-based optimization of non-intrusive ROM}
\begin{itemize}[<+->]
\item At present, the variables to be optimized are bases \(\Phi, \Psi\) and coefficients \(A, B\).
\item However, two difficulties prevent us from updating variables in standard Euclidean space of matrices.
\begin{itemize}
\item \textcolor{blue}{$\Phi$ and $\Phi Q$ gives the same decoder} if \(Q\) is any invertible r-by-r matrix.
\begin{equation}
  \label{eq:4}
  u_{r} = \Phi(\Psi^{\top}\Phi)^{-1}a = (\Phi Q)(\Psi^{\top}(\Phi Q))^{-1}a 
\end{equation}
\begin{itemize}
\item Thus, the actual variable to be trained is the \textcolor{blue}{r-dim subspace $V$ spanned by $\Phi$}.
\end{itemize}
\item Both \(\Phi\) and \(\Psi\) should have full column rank to ensure \((\Psi^{\top}\Phi)^{-1}\) exists.
\begin{itemize}
\item A natural way is to constrain \textcolor{blue}{$\Psi$ to have orthonormal columns.}
\end{itemize}
\end{itemize}
\end{itemize}
\end{frame}

\begin{frame}[label={sec:org7be6611}]{Trajectory-based optimization of non-intrusive ROM}
\begin{itemize}[<+->]
\item We now state the optimization problem as follows:
\begin{subequations}
\label{eq:5}
\begin{align}
  \underset{(V, \Psi, A, B)}{\min} J &= \sum_{m = 0}^{N_{t} - 1}\|u(t_{i}) - u_{r}(t_{i})\|^{2} \quad \text{\textcolor{blue}{(trajectory-based error)}}\\
  s.t. \quad \frac{\mathrm{d}a}{\mathrm{d}t} &= g(a) = Aa + B:(aa^{\top}), a(t_{0}) = \Psi^{\top}u(t_{0})\\
  u_{r} &= \Phi(\Psi^{\top}\Phi)^{-1}a\\
  V &= \mathrm{Range}(\Phi), \mathrm{rank}(\Phi) = r, \Psi^{\top}\Psi = I_{r}.
\end{align}
\end{subequations}
\item To solve this problem, we need to:
\begin{itemize}
\item Identify the domain (i.e. manifold) for each variable.
\item Compute the gradient of the objective on the manifold.
\item Update the variables while keeping them on their manifolds.
\end{itemize}

\item Toolbox: Pymanopt (Townsend et al. 2016)
\begin{itemize}
\item Automatically handles all these issues.
\item Allows us to update \(V\) by updating \(\Phi\).
\item Only needs user-input standard derivatives of \(J\) w.r.t. \((\Phi, \Psi, A, B)\).
\end{itemize}
\end{itemize}
\end{frame}

\section{Method}
\label{sec:org6066e78}
\begin{frame}{Contents}
\tableofcontents[currentsection, hideothersubsections]
\end{frame}

\subsection{A symmetry-reduced NiTROM (SR-NiTROM) for shift-equivariant systems}
\label{sec:orgae63797}
\begin{frame}[label={sec:org110a2ea}]{Optimization problem of SR-NiTROM}
\begin{itemize}[<+->]
\item Goal: Reduce a n-dim shift-equivariant FOM to a r-dim ROM.
\item Data: trajectory snapshots \(\{u(t_{m})\}_{m = 0}^{N_{t} - 1}\) from a shift-equivariant system. A template \(u_{0}\) for template fitting \(u\to\widehat{u}\).
\item Objective function: trajectory-based errors of \textcolor{blue}{template-fitted profile} and \textcolor{blue}{shifting amount} 
\begin{equation}
  \label{eq:6}
  J(V, \Psi, A, B) = \sum_{m = 0}^{N_{t}-1} \|\widehat{u}(t_{m}) - \widehat{u}_{r}(t_{m})\|^{2} + \beta\bigg(c(t_{m}) - c_{r}(t_{m})\bigg)^{2}
\end{equation}
\begin{itemize}
\item \(\beta\): relative weight \(\sim \|u\|^{2}/(dx)^{2}\)
\end{itemize}
\item Constraints:
\begin{itemize}
\item decoder \& encoded initial value: \(\widehat{u}_{r} = \Phi(\Psi^{\top}\Phi)^{-1}a\), \(a(t_{0}) = \Psi^{\top}\widehat{u}(t_{0})\)
\item profile equation: 
\begin{equation}
  \label{eq:7}
  \frac{\mathrm{d}a}{\mathrm{d}t} = Aa + B:(aa^{\top}) + \frac{\mathrm{d}c_{r}}{\mathrm{d}t}(\Psi^{\top}\partial_{x}\widehat{u}_{r})
\end{equation}
\item velocity equation:
\begin{equation}
  \label{eq:8}
  \frac{\mathrm{d}c_{r}}{\mathrm{d}t} = -\frac{p^{\top}a + a^{\top}Qa}{\langle\partial_{x}\widehat{u}, \partial_{x}u_{0}\rangle}
\end{equation}
\end{itemize}
\end{itemize}
\end{frame}

\begin{frame}[label={sec:orga37fe8e}]{Optimization problem of SR-NiTROM}
\begin{itemize}[<+->]
\item Unconstrained optimization problem:
\begin{align}
    & L(\Phi, \Psi, A, B, p, Q) = \sum_{m = 0}^{N_{t} - 1} L_{m}\notag\\
    & L_{m} = \|\Phi(\Psi^{\top}\Phi)^{-1}a(t_{m}) - \widehat{u}(t_{m})\|_{2}^{2} + \beta\bigg(c_{r}(t_{m}) - c(t_{m})\bigg)^{2}\notag\\
    &+ \int_{t_{0}}^{t_{m}}\lambda_{m}^{\top}(t)\bigg(\frac{\mathrm{d}a}{\mathrm{d}t} - Aa - B:(aa^{\top}) - \frac{\mathrm{d}c_{r}}{\mathrm{d}t}\Psi^{\top}\partial_{x}\Phi(\Psi^{\top}\Phi)^{-1}a\bigg)\mathrm{d}t\notag\\
    &+ \int_{t_{0}}^{t_{m}}\mu_{m}(t)\bigg(\frac{\mathrm{d}c_{r}}{\mathrm{d}t} + \frac{p^{\top}a + a^{\top}Qa}{\langle\partial_{x}\Phi(\Psi^{\top}\Phi)^{-1}a, \partial_{x}u_{0}\rangle}\bigg)\mathrm{d}t\notag\\
    &+ \lambda_{m}^{\top}(t_{0})\bigg(a(t_{0}) - \Psi^{\top}\widehat{u}(t_{0})\bigg)
\end{align}
\begin{itemize}
\item \(\lambda_{m}(t)\in\mathbb{R}^{n}, \mu_{m}(t)\in\mathbb{R}\): Lagrangian multipliers.
\item \(\partial_{a}L_{m} = 0\), \(\partial_{c_{r}}L_{m} = 0\) give adjoint equations for \(\lambda_{m}(t)\) and \(\mu_{m}(t)\).
\end{itemize}
\end{itemize}
\end{frame}

\begin{frame}[label={sec:orgc3b12b8}]{Optimization problem of SR-NiTROM}
\begin{subequations}
\begin{align}
\nabla_{\Phi}L_{m} &= \bigg(I - \Psi(\Phi^{\top}\Psi)^{-1}\Phi^{\top}\bigg)\bigg(2e(t_{m})a(t_{m})^{\top} - \partial_{x}^{\top}\Psi\int_{t_{0}}^{t_{m}}\frac{\mathrm{d}c_{r}}{\mathrm{d}t}\lambda_{m}a^{\top}\mathrm{d}t\notag\\
               &- \partial_{x}^{\top}(\partial_{x}u_{0})\int_{t_{0}}^{t_{m}}\frac{\mu_{m}(p^{\top}a + a^{\top}Qa)}{\langle\partial_{x}\Phi(\Psi^{\top}\Phi)^{-1}a, \partial_{x}u_{0}\rangle^{2}}a^{\top}\mathrm{d}t\bigg)(\Phi^{\top}\Psi)^{-1}\\
\nabla_{\Psi}L_{m} &= -2\Phi(\Psi^{\top}\Phi)^{-1}a(t_{m})e(t_{m})^{\top}\Phi(\Psi^{\top}\Phi)^{-1}\notag\\
               &- \int_{t_{0}}^{t_{m}}\partial_{x}\Phi(\Psi^{\top}\Phi)^{-1}a\lambda_{m}^{\top}\frac{\mathrm{d}c_{r}}{\mathrm{d}t}\mathrm{d}t\notag\\
               &+ \int_{t_{0}}^{t_{m}}\Phi(\Psi^{\top}\Phi)^{-1}a\lambda_{m}^{\top}\frac{\mathrm{d}c_{r}}{\mathrm{d}t}\Psi^{\top}\partial_{x}\Phi(\Psi^{\top}\Phi)^{-1}\mathrm{d}t\notag\\
               &+ \int_{t_{0}}^{t_{m}}\Phi(\Psi^{\top}\Phi)^{-1}a\mu_{m}\frac{p^{\top}a + a^{\top}Qa}{\langle\partial_{x}\Phi(\Psi^{\top}\Phi)^{-1}a, \partial_{x}u_{0}\rangle^{2}}(\partial_{x}u_{0})^{\top}\partial_{x}\Phi(\Psi^{\top}\Phi)^{-1}\mathrm{d}t\notag\\
               &- \hat{u}(t_{0})\lambda_{m}(t_{0})^{\top}        
\end{align}
\end{subequations}
\end{frame}

\begin{frame}[label={sec:org2aeb06d}]{Optimization problem of SR-NiTROM}
\begin{subequations}
\begin{align}
\nabla_{A}L_{m} &= -\int_{t_{0}}^{t_{m}}\lambda_{m}a^{\top}\mathrm{d}t\\
\nabla_{B}L_{m} &= -\int_{t_{0}}^{t_{m}}\lambda_{m}\otimes a\otimes a\mathrm{d}t\\
\nabla_{p}L_{m} &= \int_{t_{0}}^{t_{m}}\frac{\mu_{m}a}{\langle\partial_{x}\Phi(\Psi^{\top}\Phi)^{-1}a, \partial_{x}u_{0}\rangle}\mathrm{d}t\\
\nabla_{Q}L_{m} &= \int_{t_{0}}^{t_{m}}\frac{\mu_{m}aa^{\top}}{\langle\partial_{x}\Phi(\Psi^{\top}\Phi)^{-1}a, \partial_{x}u_{0}\rangle}\mathrm{d}t\\       
-\frac{\mathrm{d}\lambda_{m}}{\mathrm{d}t} &= \bigg(\frac{\partial g}{\partial a}\bigg)^{\top}\lambda_{m} - \bigg(\frac{\partial h}{\partial a}\bigg)^{\top}\mu_{m}\\
\lambda_{m}(t_{m}) &= -2(\Phi^{\top}\Psi)^{-1}\Phi^{\top}e(t_{m})
\end{align}
\end{subequations}
\end{frame}

\begin{frame}[label={sec:org1047fb6}]{Optimization problem of SR-NiTROM}
\begin{subequations}
\begin{align}
g(a) &= Aa + B:(aa^{\top}) + \frac{\mathrm{d}c_{r}}{\mathrm{d}t}\Psi^{\top}\partial_{x}\Phi(\Psi^{\top}\Phi)^{-1}a\\
h(a) &= \frac{p^{\top}a + a^{\top}Qa}{\langle\partial_{x}\Phi(\Psi^{\top}\Phi)^{-1}a, \partial_{x}u_{0}\rangle}\\
\mu_{m}(t) &= \lambda_{m}^{\top}(t)\Psi^{\top}\partial_{x}\Phi(\Psi^{\top}\Phi)^{-1}a(t) - 2\beta(c_{r}(t_{m}) - c(t_{m}))
\end{align}
\end{subequations}
\end{frame}
\end{document}
